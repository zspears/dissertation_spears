%% LyX 2.0.2 created this file.  For more info, see http://www.lyx.org/.
%% Do not edit unless you really know what you are doing.
\documentclass[english]{article}
\usepackage[T1]{fontenc}
\usepackage[latin9]{inputenc}
\usepackage{babel}
\begin{document}

\section*{Introduction}


\subsection*{Motivation}

Optimization has been the one of the main goals of computational since
the beginning of computing. Designing something to perform optimally
in all conditions is the end goal, but for the most part we simply
focus on local optimization for a single application at a time. Optimization
takes time, however. The optimization process can be a very long and
arduous task of changing the value of a single variable each time
and running a new simulation to determine what should be done with
the new variable value. This leads to many simulations for a small
benefit. This type of optimization tends to be so costly that it prohibits
very large scale optimizations simply because of run time.

With this problem in mind, adjoint based optimization was implemented.
Adjoint based optimization allows for optimization of a multitude
of control variables in a single simulation. The process simply runs
the simulation forward to the end of the interaction and then maps
back the simulation values for each step with a reverse solver to
determine the direction and magnitude that the control variables should
be moved. Adjoint based optimization is not without flaws itself.
To run an iteration of adjoint based optimization, the code must run
the entire forward simulation, while storing every step for use in
the reverse solver. This proves to be very costly for simulations
of large size in either the time or space components. For real world
problems, we often need at least one of those components to be very
large, usually both. To get valid results for use with actual problems,
we need high fidelity in the results, which means we need at least
fairly fine meshes in the simulation region leading to even larger
amounts of data to be stored for the reverse solver. 

The adjoint problem becomes intractible for the naive approach of
just simply writing out every step and then reading it back in when
necessary. For this reason, the concept of checkpointing was originally
applied. Checkpointing allows for the execution of adjoint-based optimization
with a greatly reduced need for storage space. With checkpoionting,
one runs a forward simulation and instead of storing every step, only
stores a few checkpoints from which to restart the forward solver
when the reverse solver gets to the point where it needs the steps
between them. This greatly reduces the need for storage space to run
a simulation. With improvements in how the checkpointing is done,
the possible simulations are becoming larger and longer. Thanks to
these advances, adjont based optimization has now become applicable
to very large problems. We introduce a new implementation of checkpointing
that allows for adjoint-based optiomization of fully 3D jet flows
and noise caused by them.

Jet engine noise is a problem as old as jet fighters themselves. As
the engines got and continue to get more powerful, the problem of
engine noise gets continually worse. When looking at aircraft carriers,
the problem is even larger. The solution to jet engine noise in commercial
airliners has been universally to increase the bypass ratio. Increasing
the bypass ratio allows for more cold air to be blown around the jet
engine in the cowlings around the outside. This makes for a much larger
mixing layer along with making for a much slower transition from the
ambient air to the hot and fast jet engine exhaust. This is not feasible
for military jets as the high demands on military jets mean that the
loss of possible thrust to pushing the bypass air around the engine
is completely unacceptable. Not only are there many jets on the carrier
and taking off nearly constantly, but the workers on the deck of the
carrier are exposed to the jet noise from a distance much less than
anywhere else. This creates a situation that can be extremely dangerous
for the hearing of the aircraft carrier deck workers. Under OSHA regulations,
a person working on the deck of a modern aircraft carrier could work
less than eight minutes before needing to take a full day off. The
noise is even more pervasive, given that when the workers retire from
the deck for the day, they sleep only a few floors below the source
of the noise, thus getting exposed to a portion of the noise even
in down time.

For the simulation of the real noise generated by the jets, one also
has to take into account the material properties of the jet engine
as well as the cowlings and exhaust materials. If the materials are
too flexible, the jet exhaust tip may deform and lose thrust or possibly
cause the flow to be even more over or underexpanded, leading to even
more noise. This need leads to the need of a Fluid-Structure interaction
feature in the code. 

Fluid Structure interaction is also important for the other main motivation
for this research. That motivation is protection from blast waves
created by explosions, namely improvised explosive devices. Improvised
explosive devices are a plague on the military in the current conflicts
in the middle east. With the improvements in body armor and military
intelligence along with drone warfare and bombing capabilities, more
of the members of the military are surviving the actual conflicts.
This has pushed the insurgent members of the enemy to resort to methods
of sneak attack. The sneak attacks are accomplished by either proximity
or remotely activated explosives. The blast waves from these devices
are proving to move directly through the current body armor, thus
penetrating the soldiers' bodies and creating internal hemoraging
in any cavities with air such as the lungs. The blasts are also causing
brain injuries at a rate previously unseen. This is probably due again
to the advances in military technology and the advances in medical
technology that is keeping these miliotary members who would have
died in previous conflicts. Since these soldiers are now going to
be surviving the blasts, it is the duty of science to undertake the
job of trying to prevent the blast induced brain injuries that are
now becoming a major problem for the returning soldiers.


\subsection*{Background}


\subsubsection*{CFD}

Computational Fluid Dynamics is a field of computational physics that
models the flows of fluids. This uses boundary conditions and definitions
of the gas or fluid constitution in order to simulate the interaction
of the fluids in the flow and the surrounding fluid.


\subsubsection*{LES}

Large Eddy Simulations are simulations of fluid dynamics that include
near field and far field flows of trubulent flows. They allow for
better resolution than RANS models and better computational efficiency
than DNS models. 


\subsubsection*{Unsteady Simulations}

The simulations used in the studies are unsteady in nature, meaning
that the flow does not ever reach a steady state and is in constant
flux. 


\subsubsection*{Unstructured Grids}

The code is built on unstructured grids, allowing for much more intricate
geometries than are possible for the same number of elements in a
structured code. It also allows for easier refinement in some regions
without need for global refinement. 


\subsubsection*{CSD}

Computational Structural Dynamics models the motion and reaction of
solids during collisions and in response to stresses from inside and
outside forces. 


\subsubsection*{FSI}

Fluid-Structure Interaction is a coupling of a fluid solver with a
structural sover in order to model the reaction of a solid to changes
in a fluid flow and in return the reactions of the fluid flow to the
solid and the changes in the solid. 


\subsubsection*{Adjoint-Based Optimization}

Optimization of a flow with respect to a certain trait that one would
like to minimize or maximize. Given a control surface with n points,
adjoint-based optimization can give a large step toward the optimal
solution for roughly the cost of 2-3 forward simulation compared to
n forward simulations with standard optimization techniques. Storage
and communication requirements tend to be a limiting factor for adjoint-based
optimization making it prohibitively costly without additional estimation
or other advances. 


\subsubsection*{Finite Element Method}

Finite element method description


\subsubsection*{Eulerian motion}

Fluid is described in a global reference frame. 


\subsubsection*{Lagrangian motion}

Motion is described in the frame of reference of the particle and
the motion of said particle


\subsection*{Previous Work}


\subsubsection*{Jet Engine Noise Reduction}

There have been a multitude of papers on jet engine noise reduction.
The main work that will be focused on in this paper is based on the
work by Kailasanath et al. This work is on the addition of flow disrupters
on the end of the exhaust nozzle called chevrons. These chevrons allow
for the sheer regions to be minimized by helping to add some vorticity
in the streamwise direction of the flow and thus increasing the mixing
layer. The major benefit of this is a small reduction in noise in
the aft direction, but a large reduction in the forward direction.
Namely, the chevron addition helps to reduce and eliminate the screech
from supersonic jets. Screech is the forward propogation of the shock
wave from the supersonic gasses escaping the nozzle in the subsonic
ambient air. The research also implies that the main noise-generating
region of the noise that propogates to the far field is the shear
layer noise and in order to reduce overall far field noise the goal
should be to increase the mixing layer and thus decrease the shear
layer. The pressure waves from the mixing layer tend to disipate in
the wake of the flow in much less time than that of the shear noise.
Arguably, this would be because the shear noise is escaping without
being subjected to the vorticity of the mixing layer. 

There have also been a number of suggestions for flow interrupters
inserted into the nozzle of the jet exhaust. These are shown to reduce
the noise level of the jets, but the need of military style jets to
utilize every bit of thrust available tends to make these solutions
a bit too expensive in the thrust cost to be deployable. For that
reason, the research will be looking into ways of decreasing jet engine
noise with a minimal and ideally negligible cost to thrust. 

One possible solution which could provide both a reduction in noise
and minimal cost to possibly even marginal gain in thrust is fluidic
injection at the nozzle mouth. Fluidic injection could introduce streamwise
vorticity while adding a bit of material flowing in the direction
of the jet exhaust.


\subsubsection*{Fluid Structure Interaction}

Fluid Structure interaction has been used for generations in one form
or another. It started out as one way coupling of fluid pressures
to solids and then grew to a two-way coupling of the fluid and structure.
Most of the modern fluid structure codes are using ALE solvers. ALE
stands for arbitrary Lagrangian-Eulerian solutions. This means that
the fluids are solved in eulerian form and the solids in lagrangian
form, as both are naturally done in their respective individual codes.
With ALE forms, a code can only map small deformations before it must
be regridded in order to avoid the development of nonconvex elements
which will lead to useless reults. This can introduce large time sinks
which make the ALE forms extremely costly and can introduce some extra
error which could be avoided if one could write a fluid solver and
solid solver in the same coordinates. 

Unified continuum methods attempts to do just that. Most unified continuum
methods take a strictly eulerian approaches for the fluid and the
solid. In most cases they treat the solid as eulerian for the interaction
but continually map it back to original position in order to calculate
the standard lagrangian for of the solids and then map back to eulerian
for the interface. The interfact is typically tracked from the movement
of the solid and is kept in a continuum method such that a small amount
of area in the boundary region between the solid and fluid is treated
as a mixture of the two. This leads to a semi-stagnation of the fluid
around the solid and can lead to some non-physical results. 
\end{document}
